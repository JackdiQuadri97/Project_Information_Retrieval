\section{Introduction}
\label{sec:introduction}

Before the era of the internet, information storage and retrieval systems mostly used by professional people for instance: medical research, in libraries, by governmental organizations and archives. That made access to such information a hard process especially for non-search experts. Recently with the fast increase in the number of data and information available online the importance of the search engines raised dramatically. Nowadays people use search engines for instance to locate and buy goods, to choose a vacation destination, to select a medical treatment or to find background information on candidates of an election. Search engines transitioned from being a searcher's tool for information to a tool for building opinions and making major decisions. All those factors combined together make argument retrieval systems a necessary to make a real impact in the industry, in order to enhance the information retrieval field. For this purpose we developed this arguments retrieval system as a Java project.


The paper is organized as follows: Section~\ref{sec:methodology} describes our approach; Section~\ref{sec:setup} explains our experimental setup; Section~\ref{sec:results} discusses our main findings; finally, Section~\ref{sec:conclusion} draws some conclusions and outlooks for future work.